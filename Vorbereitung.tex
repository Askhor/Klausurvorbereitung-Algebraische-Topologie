% !TeX spellcheck = de_DE

\documentclass{article}

\usepackage[utf8]{inputenc}
\usepackage{unicode-helper}
\usepackage{amsmath}
\usepackage{amssymb}
\usepackage{amsfonts}
\usepackage{microtype}
\usepackage{tikz-cd}
\usepackage[english]{babel}
\usepackage{amsthm}
\usepackage{mathtools}
\usepackage{braket}
\usepackage{csquotes}
\usepackage[backend=biber,style=numeric]{biblatex}
\usepackage[pdftex]{hyperref}
\usepackage{cleveref}

\newcommand{\titlevar}{Klausurvorbereitung Algebraische Topologie}
\newcommand{\authorvar}{Günthner}
\newcommand{\datevar}{Winter 2024}
\title{\titlevar}
\author{\authorvar}
\date{\datevar}
\hypersetup{
	pdftitle=\titlevar,
	pdfauthor=\authorvar,
	pdfcreationdate=\datevar,
}
\setlength{\parindent}{0pt}

\newlength{\halflength}
\setlength{\halflength}{100pt}

\newcommand{\card}[1]{\#{#1}}
\newcommand{\inv}{^{-1}}
\newcommand{\ringunits}[1]{{#1}^{\mathclap{\rule{2.5pt}{0pt}\times}\rule{0.2em}{0pt}}}
\newcommand{\ringunitsb}[1]{\ringunits{\left({#1}{\vphantom{p^k}}\right)}}
\newcommand{\ordgroup}[1]{\ord_{\rule{0pt}{9pt}\mathclap{#1}}}
\newcommand{\ordadd}[1]{\ordgroup{ℤ/{#1}ℤ}}
\newcommand{\ordmult}[1]{\ord_{\rule{0pt}{8pt}{
			\settowidth{\halflength}{$\scriptstyle \left({ℤ/{#1}ℤ}{\vphantom{p^k}}\right)$}
			\mathrlap{\rule{-0.5 \halflength}{0pt}\ringunitsb{ℤ/{#1}ℤ}}
}}}
\newcommand{\ordker}[1]{\ordgroup{\ker(\mathrlap{#1)}}}
\newcommand{\frakB}{\mathfrak{B}}
\newcommand{\frakc}{\mathfrak{c}}
\newcommand{\bigbarn}[1]{\big({#1}\big)}
\newcommand{\Bigbarn}[1]{\Big({#1}\Big)}
\newcommand{\spospart}[1]{ {(#1)} _ {\rule{1.5pt}{0pt}\mathclap +} }
\newcommand{\pospart}[1]{{ {\bigbarn{#1}} _ {\mathclap +} }}
\newcommand{\tritem}{\item[$\blacktriangleright$]}
\newcommand{\gspan}[1]{<#1>}
\DeclareMathOperator{\ordb}{ord}
\DeclareMathOperator{\img}{img}
\newcommand{\ord}{\mathop{\ordb}\limits}
\newcommand{\powerset}{\mathcal{P}}
\DeclareMathOperator{\lcm}{lcm}
\DeclareMathOperator{\id}{id}
\newenvironment{pg}{
	
}{
	
	\medskip
	
}
\newcommand{\maybeequal}{\stackrel{?}{=}}
\newcommand{\mapdefinition}[5]{
	\begin{center}
		\begin{tabular}{llll}
			$#1:$ 	&	$#2$ & $→$ & $#3$ 	\\
			&	$#4$ & $↦$ & $#5$	\\
		\end{tabular}
	\end{center}
}
\newcommand{\padlr}[2]{\rule{#2}{0pt} {#1} \rule{#2}{0pt}}
\newcommand{\buildset}[2]{\set{{#1} \ : \ {#2}}}

\newtheorem{definition}{Definition}
\newtheorem{lemma}{Lemma}
\newtheorem{theorem}{Theorem}
\newtheorem{conjecture}{Conjecture}

\crefname{lemma}{Lemma}{Lemmata}
\crefname{equation}{equation}{equations}
\crefname{theorem}{Theorem}{Theorems}
\crefname{definition}{Definition}{Definitions}

\begin{document}
	\maketitle
	
	\section{Singuläre Homologie}

	% https://q.uiver.app/#q=WzAsMyxbMSwwLCJDX2soWCkiXSxbMiwwLCJDX3trLTF9KFgpIl0sWzAsMCwiQ197aysxfShYKSJdLFswLDEsImRfayJdLFsyLDAsImRfe2srMX0iXV0=
	\[\begin{tikzcd}
		{C_{k+1}(X)} & {C_k(X)} & {C_{k-1}(X)}
		\arrow["{d_{k+1}}", from=1-1, to=1-2]
		\arrow["{d_k}", from=1-2, to=1-3]
	\end{tikzcd}\]
	
	\begin{definition}
		\begin{equation*}
			H_k(X) = \ker(d_k) / \img(d_{k+1})
		\end{equation*}
	\end{definition}
	
	\subsection{Homologie vom Punkt}
	
	\begin{equation*}
		C_k(*) = ℝ · \text{konstante Abb.}
	\end{equation*}
	\begin{equation*}
		\img(d_k) = \begin{cases}
			ℝ & \text{falls $k$ gerade}\\
			0 & \text{falls $k$ ungerade}\\
		\end{cases}	
	\end{equation*}
	\begin{equation*}
		\ker(d_k) = \begin{cases}
			ℝ & \text{falls $k$ ungerade}\\
			0 & \text{falls $k$ gerade}\\
		\end{cases}
	\end{equation*}
	Für $k≥1$:
	\begin{equation*}
		H_k(*) = \ker(d_k) / \img(d_{k+1}) = 0
	\end{equation*}
	Für $k=0$:
	\begin{equation*}
		H_0(*) = \ker(d_k) / \img(d_{k+1}) = 0/0 = 0
	\end{equation*}
	
	\subsection{Homotopie-Invarianz}
	
	Seien $X, Y$ topologische Räume
\end{document}