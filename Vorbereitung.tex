% !TeX spellcheck = de_DE

\documentclass{article}
\newcommand{\documentlanguage}{ngerman}

\usepackage[utf8]{inputenc}
\usepackage[\documentlanguage]{babel}
\usepackage{unicode-helper}
\usepackage{amsthm}
\usepackage{amsmath}
\usepackage{amssymb}
\usepackage{amsfonts}
\usepackage{microtype}
\usepackage{tikz-cd}
\usepackage[T1]{fontenc}
\usepackage{mathtools}
\usepackage{braket}
%\usepackage{enumitem}
\usepackage{csquotes}
\usepackage[pdftex]{hyperref}
\usepackage[\documentlanguage]{cleveref}

\newcommand{\titlevar}{Klausurvorbereitung Algebraische Topologie}
\newcommand{\authorvar}{Günthner}
\newcommand{\datevar}{Winter 2024}
\title{\titlevar}
\author{\authorvar}
\date{\datevar}
\hypersetup{
	pdftitle=\titlevar,
	pdfauthor=\authorvar,
	pdfcreationdate=\datevar,
}
\setlength{\parindent}{0pt}

%\setlist{label=$\blacktriangleright$}

\newlength{\halflength}
\setlength{\halflength}{100pt}

\newcommand{\card}[1]{\#{#1}}
\newcommand{\inv}{^{-1}}
\newcommand{\ringunits}[1]{{#1}^{\mathclap{\rule{2.5pt}{0pt}\times}\rule{0.2em}{0pt}}}
\newcommand{\ringunitsb}[1]{\ringunits{\left({#1}{\vphantom{p^k}}\right)}}
\newcommand{\ordgroup}[1]{\ord_{\rule{0pt}{9pt}\mathclap{#1}}}
\newcommand{\ordadd}[1]{\ordgroup{ℤ/{#1}ℤ}}
\newcommand{\ordmult}[1]{\ord_{\rule{0pt}{8pt}{
			\settowidth{\halflength}{$\scriptstyle \left({ℤ/{#1}ℤ}{\vphantom{p^k}}\right)$}
			\mathrlap{\rule{-0.5 \halflength}{0pt}\ringunitsb{ℤ/{#1}ℤ}}
}}}
\newcommand{\ordker}[1]{\ordgroup{\ker(\mathrlap{#1)}}}
\newcommand{\frakB}{\mathfrak{B}}
\newcommand{\frakc}{\mathfrak{c}}
\newcommand{\bigbarn}[1]{\big({#1}\big)}
\newcommand{\Bigbarn}[1]{\Big({#1}\Big)}
\newcommand{\spospart}[1]{ {(#1)} _ {\rule{1.5pt}{0pt}\mathclap +} }
\newcommand{\pospart}[1]{{ {\bigbarn{#1}} _ {\mathclap +} }}
\newcommand{\tritem}{\item[$\blacktriangleright$]}
\newcommand{\gspan}[1]{<#1>}
\DeclareMathOperator{\ordb}{ord}
\DeclareMathOperator{\img}{img}
\newcommand{\ord}{\mathop{\ordb}\limits}
\newcommand{\powerset}{\mathcal{P}}
\DeclareMathOperator{\lcm}{lcm}
\DeclareMathOperator{\id}{id}
\newcommand{\maybeequal}{\stackrel{?}{=}}
\newcommand{\mapdefinition}[5]{
	\begin{center}
		\begin{tabular}{llll}
			$#1:$ 	&	$#2$ & $→$ & $#3$ 	\\
			&	$#4$ & $↦$ & $#5$	\\
		\end{tabular}
	\end{center}
}
\newcommand{\padlr}[2]{\rule{#2}{0pt} {#1} \rule{#2}{0pt}}
\newcommand{\buildset}[2]{\set{{#1} \ : \ {#2}}}
\DeclareMathOperator{\lspan}{span}

\newtheorem{definition}{Definition}
\newtheorem{lemma}{Lemma}
\newtheorem{theorem}{Theorem}
\newtheorem{conjecture}{Conjecture}

\crefname{lemma}{Lemma}{Lemmata}
\crefname{equation}{equation}{equations}
\crefname{theorem}{Theorem}{Theorems}
\crefname{definition}{Definition}{Definitions}

\begin{document}
	\shorthandoff{"}
	\maketitle

	\tableofcontents
	
	\section{Singuläre Homologie}

	\subsection{Definition}

	\begin{definition}[Simplex]
		\begin{equation*}
			Δ^k = \set{(a_i) ϵ ℝ^{k+1} : 0≤a_i≤1 \text{ und } Σ_{i=1}^{n} a_i = 1}
		\end{equation*}
	\end{definition}

	\begin{definition}[Kettenkomplex]
		\begin{equation*}
			C_k(X) = \lspan\set{Δ^k → X \text{ stetig}}
		\end{equation*}
	\end{definition}

	\begin{definition}[Randabbildung]
		Wir definieren $d$ als linear und dann nur Basisvektoren $φ$:
		\begin{equation*}
			\begin{split}
				d_k: C_k &→ C_{k-1} \\
				φ &↦ Σ_{i=1}^{k} (-1)^i \underbrace{(p_i)^*}_{\text{pullback}} φ
			\end{split}
		\end{equation*}

		mit 
		
		\begin{equation*}
			\begin{split}
				p_i: Δ_{k-1} &→ Δ_k \\
				(x_j) &↦ (x_1, \dots, x_{j-1}, 0, x_{j}, \dots, x_{k-1})
			\end{split}
		\end{equation*}
	\end{definition}

	% https://q.uiver.app/#q=WzAsMyxbMSwwLCJDX2soWCkiXSxbMiwwLCJDX3trLTF9KFgpIl0sWzAsMCwiQ197aysxfShYKSJdLFswLDEsImRfayJdLFsyLDAsImRfe2srMX0iXV0=
	\[\begin{tikzcd}
		{C_{k+1}(X)} & {C_k(X)} & {C_{k-1}(X)}
		\arrow["{d_{k+1}}", from=1-1, to=1-2]
		\arrow["{d_k}", from=1-2, to=1-3]
	\end{tikzcd}\]
	
	\begin{definition}[Singuläre Homologie]
		\begin{equation*}
			H_k(X) = \ker(d_k) / \img(d_{k+1})
		\end{equation*}
	\end{definition}
	
	\subsection{Homologie vom Punkt}
	
	\begin{equation*}
		C_k(*) = ℝ · \text{konstante Abb.}
	\end{equation*}
	\begin{equation*}
		\img(d_k) = \begin{cases}
			ℝ & \text{falls $k$ gerade}\\
			0 & \text{falls $k$ ungerade}\\
		\end{cases}	
	\end{equation*}
	\begin{equation*}
		\ker(d_k) = \begin{cases}
			ℝ & \text{falls $k$ ungerade}\\
			0 & \text{falls $k$ gerade}\\
		\end{cases}
	\end{equation*}
	Für $k≥1$:
	\begin{equation*}
		H_k(*) = \ker(d_k) / \img(d_{k+1}) = 0
	\end{equation*}
	Für $k=0$:
	\begin{equation*}
		H_0(*) = \ker(d_k) / \img(d_{k+1}) = 0/0 = 0
	\end{equation*}
	
	\subsection{Homotopie-Invarianz}
	
	Seien $X, Y$ topologische Räume und $g: X→Y, h: Y→X$ mit
	\begin{equation*}
		g ∘ h \sim \id \text{ and } h∘g \sim \id
	\end{equation*}
	
	Zeigen wir, dass
	\begin{equation*}
		g∘h \sim \id ⇒ (g∘h)_* = \id
	\end{equation*}
	
	Sei hierfür $H: [0,1] × X → X$ eine Homotopie zwischen $H(0)$ und $H(1)$, dann erhalten wir durch den Prismenoperator eine Kettenhomotopie.
	
	% https://q.uiver.app/#q=WzAsNixbMCwwLCJDKEEpIl0sWzEsMCwiQyhBKSJdLFsyLDAsIkMoQSkiXSxbMCwxLCJDKEIpIl0sWzEsMSwiQyhCKSJdLFsyLDEsIkMoQikiXSxbMCwxLCJkIl0sWzEsMiwiZCJdLFszLDQsImQiXSxbNCw1LCJkIl0sWzEsMywiSCIsMSx7ImNvbG91ciI6WzAsNjAsNjBdfSxbMCw2MCw2MCwxXV0sWzIsNCwiSCIsMSx7ImNvbG91ciI6WzAsNjAsNjBdfSxbMCw2MCw2MCwxXV0sWzEsNCwiZixnIl0sWzIsNSwiZixnIl0sWzAsMywiZixnIl1d
	\[\begin{tikzcd}
		{C(A)} & {C(A)} & {C(A)} \\
		{C(B)} & {C(B)} & {C(B)}
		\arrow["d", from=1-1, to=1-2]
		\arrow["{f,g}", from=1-1, to=2-1]
		\arrow["d", from=1-2, to=1-3]
		\arrow["H"{description}, color={rgb,255:red,214;green,92;blue,92}, from=1-2, to=2-1]
		\arrow["{f,g}", from=1-2, to=2-2]
		\arrow["H"{description}, color={rgb,255:red,214;green,92;blue,92}, from=1-3, to=2-2]
		\arrow["{f,g}", from=1-3, to=2-3]
		\arrow["d", from=2-1, to=2-2]
		\arrow["d", from=2-2, to=2-3]
	\end{tikzcd}\]
	
	Mit $f-g = dH + Hd$
	
	\subsection{Mayer-Vietoris}
	\label{sec:mvb1}
	
	Sei $U ∪ V$ ein topologischer Raum mit $U,V$ offen. Versuchen wir folgende exakte Sequenz zu zeigen:
	% https://q.uiver.app/#q=WzAsNixbMiwwLCJIX2soVeKIqVYpIl0sWzMsMCwiSF9rKFUpIFxcb3BsdXMgSF9rKFYpIl0sWzQsMCwiSF9rKFXiiKpWKSJdLFsxLDAsIkhfe2srMX0oVeKIqlYpIl0sWzAsMF0sWzUsMF0sWzAsMV0sWzEsMl0sWzMsMF0sWzQsMywiXFxjZG90cyIsMSx7ImxhYmVsX3Bvc2l0aW9uIjoxMH1dLFsyLDUsIlxcY2RvdHMiLDEseyJsYWJlbF9wb3NpdGlvbiI6OTB9XV0=
	\[\begin{tikzcd}
		{} & {H_{k+1}(U∪V)} & {H_k(U∩V)} & {H_k(U) \oplus H_k(V)} & {H_k(U∪V)} & {}
		\arrow["\cdots"{description, pos=0.1}, from=1-1, to=1-2]
		\arrow[from=1-2, to=1-3]
		\arrow[from=1-3, to=1-4]
		\arrow[from=1-4, to=1-5]
		\arrow["\cdots"{description, pos=0.9}, from=1-5, to=1-6]
	\end{tikzcd}\]
	Hierfür werden wir folgende isomorphe Sequenz zeigen:
	% https://q.uiver.app/#q=WzAsNixbMiwwLCJIX2soVeKIqVYpIl0sWzMsMCwiSF9rKFUpIFxcb3BsdXMgSF9rKFYpIl0sWzQsMCwiSF9rKFUrVikiXSxbMSwwLCJIX3trKzF9KFUrVikiXSxbMCwwXSxbNSwwXSxbMCwxXSxbMSwyXSxbMywwXSxbNCwzLCJcXGNkb3RzIiwxLHsibGFiZWxfcG9zaXRpb24iOjEwfV0sWzIsNSwiXFxjZG90cyIsMSx7ImxhYmVsX3Bvc2l0aW9uIjo5MH1dXQ==
	\[\begin{tikzcd}
		{} & {H_{k+1}(U+V)} & {H_k(U∩V)} & {H_k(U) \oplus H_k(V)} & {H_k(U+V)} & {}
		\arrow["\cdots"{description, pos=0.1}, from=1-1, to=1-2]
		\arrow[from=1-2, to=1-3]
		\arrow[from=1-3, to=1-4]
		\arrow[from=1-4, to=1-5]
		\arrow["\cdots"{description, pos=0.9}, from=1-5, to=1-6]
	\end{tikzcd}\]
	Dass können wir unter Verwendung des Schlangenlemmas (?) und folgender kurzen exakten Sequenz zeigen:
	% https://q.uiver.app/#q=WzAsNSxbMSwwLCJDX2soVeKIqVYpIl0sWzIsMCwiQ19rKFUpIFxcb3BsdXMgQ19rKFYpIl0sWzMsMCwiQ19rKFUrVikiXSxbMCwwLCIwIl0sWzQsMCwiMCJdLFswLDFdLFsxLDJdLFszLDBdLFsyLDRdXQ==
	\[\begin{tikzcd}
		0 & {C_k(U∩V)} & {C_k(U) \oplus C_k(V)} & {C_k(U+V)} & 0
		\arrow[from=1-1, to=1-2] 
		\arrow[from=1-2, to=1-3]
		\arrow[from=1-3, to=1-4]
		\arrow[from=1-4, to=1-5]
	\end{tikzcd}\]
	
	\section{De-Rahm Cohomologie}
	
	\subsection{Poincaré Lemma}
	
	$U$ sternförmig und offen in $ℝ^n$, zz. für $ω ϵ Ω_k(U)$ mit $dω = 0$:
	\begin{equation*}
		∃ ηϵΩ_{k+1}(U) \text{ mit } dη = ω
	\end{equation*}
	
	Definieren wir 
	\begin{equation*}
		η=ι_X ∫\limits_{-∞}^0 (φ_X^t)^*(ω) dt
	\end{equation*}
	Nun erhalten wir für $dη$:
	\begin{align*}
			dη &= dι_X ∫\limits_{-∞}^0 (φ_X^t)*(ω) dt = (L_X - ι_Xd) ∫\limits_{-∞}^0 (φ_X^t)^*(ω) dt &\\
			&= \frac{d}{ds}|_{s=0}  ∫\limits_{-∞}^0 (φ_X^t)^*(ω) dt - ι_Xd ∫\limits_{-∞}^0 (φ_X^t)^*(ω) dt &\\
			&= ∫\limits_{-∞}^0 \frac{d}{ds}|_{s=0} (φ_X^s)^* (φ_X^t)^*(ω) dt - ι_X ∫\limits_{-∞}^0 (φ_X^t)^*(dω) dt &\\
			&= ∫\limits_{-∞}^0 \frac{d}{ds}|_{s=t} (φ_X^s)^*(ω) dt - 0 &\\
			&= ω - \lim_{x→-∞} (φ_X^x)^*(ω) = ω - 0 &= ω
	\end{align*}
	
	\subsection{Mayer-Vietoris}
	
	Vorgehen: gleich wie in $\cref{sec:mvb1}$
	
	Wir versuchen also Exaktheit von folgender Kette zu zeigen:
	% https://q.uiver.app/#q=WzAsNixbMCwwLCLOqV9rKFXiiKpWKSJdLFsxLDAsIs6pX2soVSkgXFxvcGx1cyDOqV9rKFYpIl0sWzIsMCwizqlfayhV4oipVikiXSxbMCwxLCLPiSJdLFsxLDEsIijPiSwgLc+JKSBcXCBcXCAoz4lfVSwgz4lfVikiXSxbMiwxLCLPiV9VICsgz4lfViJdLFswLDEsIs6xIl0sWzEsMiwizrIiXSxbMyw0LCIiLDAseyJzdHlsZSI6eyJ0YWlsIjp7Im5hbWUiOiJtYXBzIHRvIn19fV0sWzQsNSwiIiwwLHsic3R5bGUiOnsidGFpbCI6eyJuYW1lIjoibWFwcyB0byJ9fX1dXQ==
	\[\begin{tikzcd}
		{Ω_k(U∪V)} & {Ω_k(U) \oplus Ω_k(V)} & {Ω_k(U∩V)} \\
		{ω} & {(ω, -ω) \ \ (ω_U, ω_V)} & {ω_U + ω_V}
		\arrow["{α}", from=1-1, to=1-2]
		\arrow["{β}", from=1-2, to=1-3]
		\arrow[maps to, from=2-1, to=2-2]
		\arrow[maps to, from=2-2, to=2-3]
	\end{tikzcd}\]	
	
	$α$ ist injektiv, denn sei $ω$ im Kern von $α$, dann folgt schon, dass $ω$ null auf $U$, $V$ und somit auch auf $U∪V$ ist.

	TODO

	\subsection{Homologie vom Torus}

	\newcommand{\torus}{\mathbb{T}^n}
	Definieren wir den $n$-Torus als $\torus = ℝ^n/ℤ^n$. Wir wollen nun mit Hilfe von De-Rahm Cohomologie zeigen, dass $b_k(\mathbb{T}^n) = \binom{n}{k}$. Dass machen wir in drei Schritten:
	\begin{enumerate}
		\item Wenn wir Differentialformen verschieben, dann bleibt die Homologieklasse gleich
		\item Wenn wir dann eine Differentialform mittlen, dann bleibt die Homologieklasse wieder gleich
		\item Nun gibt es für jede Homologieklasse immer einen konstanten Repräsentanten und wir können nun die Homologie mit dem Zielraum der Differentialformen identifizieren: $Λ^k(ℝ^n)$. Und das hat Dimension $\binom{n}{k}$
	\end{enumerate}

	\subsubsection{Verschieben lässt Homologieklasse gleich}

	Was wollen wir zeigen? Sei $v ϵ \torus$, dann definieren wir $φ_v: \torus → \torus, x ↦ x + v$, nun hätten wir gerne, dass für $ωϵH^k(\torus)$ gilt: $[ω] = [(φ_v)^* ω]$.

	\medskip

	Hierfür definieren wir das Vektorfeld $v$, dass einfach überall konstant gleich $v$ ist. Der Fluss darüber ist dann einfach $φ_v^t$.

	\medskip

	Nun, wir wollen zeigen, dass $[ω] = [(φ_v^1)^* ω]$, also dass $(φ_v¹)^* ω - (φ_v⁰)^* ω ϵ \img d$.

	\medskip

	Als erstes erkennen wir, dass diese Differenz das gleiche ist wie das Integral über lokale Differenzen entlang einem bestimmten Pfad:
	\begin{equation*}
		\begin{split}
			&\int_0^1 L_v (φ_v^t)^* ω dt \\
			&= \int_0^1 (dι_v + ι_vd) (φ_v^t)^* ω dt \\
			&= d \int_0^1 ι_v (φ_v^t)^* ω dt + \underbrace{\int_0^1 ι_v (φ_v^t)^* dw dt}_0
		\end{split}
	\end{equation*}

	\qed

	\subsubsection{Mitteln lässt Homologieklasse gleich}

	Definieren wir Mitteln in eine Richtung:
	\begin{equation*}
		\begin{split}
			M_i: Ω^k(\torus) &→ Ω^k(\torus) \\
			ω &↦ \int_0^1 (φ_{x_i}^t)^* ω dt
		\end{split}
	\end{equation*}

	Nun können wir insgesamtes Mitteln können wir definieren als
	\begin{equation*}
		M = M_1 ∘ M_2 ∘ \dots ∘ M_n
	\end{equation*}

	\medskip

	$M_i$ lässt die Homologieklasse gleich warum? Naja, es ist ein Integral über etwas wie $(φ_{x_i}^t)^* ω$ und die haben ja alle die gleiche Homologieklasse. Also integrieren wir über Elemente eines Untervektorraums und das Resultat ist dann auch in diesem Vektorraum.

	\medskip

	Nun lässt $M$ die Homologieklasse gleich, da es die Verknüpfung von Abbildungen mit dieser Eigenschaft.

	\qed

	\subsubsection{Isomorphie \texorpdfstring{$H^k(\torus) \cong Λ^k(ℝ^n)$}{H^k(T^n) = Λ^k(ℝ^n)}}

	Nehmen wir ein Element $[ω] ϵ H^k(\torus)$. Nun nehmen wir $[ω] = [M(ω)]$, d.h. jede Homologieklasse hat einen gemittelten Repräsentanten, also einen konstanten. D.h. wir können $H^k(\torus)$ mit $Ω_c^k(\torus)$—den konstanten Differentialformen—identifizieren.

	\medskip

	Die konstanten Differentialformen können wir natürlich mit $Λ^k(ℝ^n)$ identifizieren, indem wir die Differentialform irgendwo ausowerten $M(ω)(x)$.

	\qed
\end{document}
